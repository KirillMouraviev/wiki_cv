\documentclass[]{article}
\usepackage{cite}
\usepackage{indentfirst}
\usepackage{amsfonts}
\usepackage{amsmath}
\usepackage{graphicx}

\usepackage[T2A]{fontenc}
\usepackage[utf8]{inputenc}
\usepackage[russian]{babel}

% ----  ADDED IN OVERLEAF ---

\usepackage{geometry}
\geometry{
 a4paper,
 total={170mm,257mm},
 left=20mm,
 top=20mm,
}

\usepackage{titlesec}
\titlelabel{\thetitle.\quad}

% ----------------------------

\let\v\mathbf

%opening
\title{Обработка трехмерных облаков точек с помощью нейронных сетей}
\author{Кирилл Муравьев}

\begin{document}

\maketitle

В настоящее время трехмерные облака точек (point clouds) широко используются в робототехнике и компьютерном моделировании. Например, основными данными для управления и навигации беспилотных автомобилей являются облака точек, поступающие с лидаров. Также облако точек является одним из самых удобных форматов для воссоздания трехмерной модели объекта. Часто возникают задачи классификации или сегментации набора точек в трехмерном пространстве. Для решения подобных задач нужна высокоуровневая обработка точек с учетом геометрии, цвета и других характеристик. До появления нейронных сетей задачи классификации объектов по облакам точек решались классическими методами, основанными на различных способах вычисления дескрипторов по окрестностям точек, например, \cite{lu2014recognizing}, \cite{guo20143d}. Однако такие методы обладают недостаточно высоким качеством работы, поэтому в последнее время для обработки облаков точек применяются в основном методы, основанные на нейронных сетях.

В данном тексте рассмотрим два метода генерации признаков по набору точек с помощью нейронных сетей: пространственный и графовый. В пространственном методе облака точек переводятся бинарные сетки заполненности (occupancy grid), по которым с помощью трехмерных сверток вычисляются целевые признаки. В графовом методе на облаке точек задается структура графа, по которому вычисляются целевые признаки с помощью графовых сверточных сетей.

\section{Постановка задачи}

Дан набор точек в трехмерном пространстве: 
$$P \in \mathbb{R}^{n \times 3} = \{p_i = (x_i, y_i, z_i), i = 1, \dots, n\}$$
В задаче классификации облаков точек нужно восстановить принадлежность объекта к одному из $K$ заданных классов:
$$f(P) \sim c \in \{1, \dots, K\}$$

В задаче сегментации облаков точек нужно восстановить принадлежность каждой точки к одному из $K$ классов:
$$f(P) \sim \{c_1, \dots, c_n\}, c_i \in \{1, \dots, K\}$$

В задаче детекции объекта нужно найти ограничивающий параллелепипед (bounding box) минимального размера, содержащий объект определенного класса:
$$f(P) \sim (x_{min}, x_{max}, y_{min}, y_{max}, z_{min}, z_{max})$$
$$\forall i: c_i = c \hookrightarrow x_i \in [x_{min}, x_{max}], y_i \in [y_{min}, y_{max}], z_i \in [z_{min}, z_{max}]$$

\section{Пространственный метод}

\subsection{Описание метода}

В пространственном методе проводится 
дискретизация облаков точек - все пространство разбивается трехмерной прямоугольной сеткой на ячейки определенного размера. Если в ячейке есть точка из облака, она помечается как занятая, иначе - как свободная. Так получается трехмерная дискретная структура, над которой можно выполнять обычные свертки:

$$X \in \{0, 1\}^{H \times W \times D};$$
$$X_{i,j,k} = \begin{cases}
0,\ \ \exists p_i = (x_i, y_i, z_i): x_i \in [ih,(i+1)h], y_i \in [jh,(j+1)h], z_i \in [kh,(k+1)h]\\
1,\ \ \ \ \ \ \ \ \ \ \ \ \ \ \ \ \ \ \ \ \ \ \ \ \  \ \ \ \ \ \ \ \ \ \ \ \ \ \ \ \ \ \ \ \ \ \ \ \ \ \ \ \ \ \  \ \ \ \ \ \ \ \ \ \ \ \ \ \ \ \ \ \ \ \ \ \ \ \ \ \ \ \ \ \text{otherwise}
\end{cases},$$
где $h$ - шаг сетки. Также можно вместо флага заполненности помещать в ячейки сетки другие признаки, например, цвет точки, содержащейся в ячейке.

Трехмерная свертка не сильно отличается от двумерной. Она выглядит следующим образом:
$$(K*X)_{i,j,k} = \sum\limits_{i'=-n}^n \sum\limits_{j'=-m}^m \sum\limits_{k'=-q}^q X_{i-i',j-j',k-k'} K_{i'+m,j'+n,k'+q}$$

В общем виде сверточную нейронную сеть можно представить следующим образом:
$$f(X) = (f_1 \circ f_n \circ \dots \circ f_l)(x),$$
$$f_i(X) = \sigma_i(K_i * X),$$
где функции $\sigma_i$ задают различные нелинейности и нормализации.

\subsection{Современные нейросетевые архитектуры}

На сегодняшний день существует множество нейросетевых архитектур для обработки облаков точек с помощью пространственного метода. Рассмотрим некоторые из них. 

Одна из таких архитектур представлена в работе \cite{li20173d}. В этой работе решается задача детекции автомобилей в облаке точек. Облако точек дискретизуется трехмерной сеткой с шагом 10 см. Сеть состоит из трех сверточных слоев с размером ядра 4, и двух слоев развертки (deconvolution). Первый слой развертки предсказывает наличие автомобиля в клетке пространства, а второй - размер параллелепипеда, содержащего автомобиль. Эксперименты в этой работе проводились на коллекции данных KITTI \cite{geiger2013vision}. Перекрытие предсказанного положения автомобиля с истинным составило от 70\% до 90\% на разных сценах.

Еще одна архитектура для обработки облаков точек представлена в работе \cite{huang2016point}. В ней решается задача сегментации облаков точек. Облако точек дискретизуется сеткой с шагом 30 см и затем подается на вход сети, состоящей из трех трехмерных сверток с ядром размера 5 и трехмерной субдискретизацией (3D max pooling) и одного полносвязного слоя для предсказания класса точки. Эксперименты проводились на коллекции облаков точек, собранных с лидаров в городской среде. Точность сегментации при 7 классах достигла 93\%.

\subsection{Преимущества и недостатки метода}

Основное преимущество пространственного метода - наличие пространственной структуры, которая позволяет извлекать признаки с помощью сверточных нейронных сетей. Трехмерные свертки можно эффективно вычислять на многопоточном компьютере или на видеокарте. Использование большого количества сверток позволяет добиться высокого качества классификации, сегментации и детекции объектов по облакам точек.

Еще одно несомненное преимущество пространственного метода - легкость реализации. Эффективное параллельное вычисление трехмерных сверток реализовано в большинстве фреймворков глубокого обучения. Дискретизация облака точек также является простым алгоритмом и может быть легко и эффективно реализована.

К недостаткам пространственного метода следует отнести высокую ресурсоемкость, которая неизбежно возникает при использовании трехмерных матриц и сверток на них. Трехмерные свертки требуют не только болших вычислительных затрат, но и больших объемов памяти для хранения весов и промежуточных тензоров. Поэтому применение нейронных сетей с пространственным методом на маломощных бортовых вычислителях может быть затруднено.

\section{Графовый метод}

\subsection{Описание метода}

В графовом методе с помощью определенного алгоритма на облако точек наносится структура графа. Вершинами графа являются точки исходного облака. Ребра задаются чаще всего по расстояниям между точками. Например, один из вариантов задания ребер - провести от каждой точки ребро к ее $k$ ближайшим соседям в исходном пространстве.

Свертка на графах задается с помощью реберных функций - каждому ребру графа сопоставляется функция, имеющая настраиваемые параметры. Значением свертки в вершине является агрегация (сумма, максимум или другая) реберных функций по всем ребрам, входящим в эту вершину:
$$x'_j = \sum\limits_{(i,j) \in E} h_{\theta}(x_i, x_j) \text{ или } x'_j = \max\limits_{(i,j) \in E} h_{\theta}(x_i, x_j)$$

\subsection{Современные нейросетевые архитектуры}

Одной из самых популярных нейросетей для сегментации облаков точек является архитектура PointNet \cite{qi2017pointnet}. В этой архитектуре к исходным точкам применяется серия матричных преобразований и полносвязных слоев, а затем - агрегация с помощью global max pooling. Эту агрегацию можно интерпретировать как свертку на графе без ребер, где функция $h_{\theta}(i, j)$ зависит только от вершины $i$. Несмотря на простоту архитектуры, PointNet обладает довольно высоким качеством работы - точность сегментации на датасете ModelNet40 \cite{wu20153d} составила 89.2\%.

Еще более высокого качества сегментации удалось добиться с помощью архитектуры PointNet++ \cite{qi2017pointnet++}, которая является усовершенствованной версией архитектуры PointNet. Точность сегментации на датасете ModelNet40 составила 90.7\%. В этой архитектуре используются графы, в которых точки объединены в кластеры - компоненты связности. Каждая точка в кластере соединена с центроидом. Агрегация при свертке осуществляется максимизацией.

Еще одна работа по сегментации облаков точек с помощью сверток на графах - PCNN \cite{atzmon2018point}. В этой работе за счет применения операций Extension и Restriction, а также свертки с RBF-ядром удалось добиться точности сегментации в 93.2\%.

Одна из последних работ (на декабрь 2019 года), посвященная задаче сегментации облаков точек - это EdgeConv \cite{wang2019dynamic}. В архитектуре, представленной в этой работе, на точках строится так называемый kNN-граф: точка соединяется ребрами с $k$ ближайшими к ней точками в исходном пространстве. В качестве реберной функции $h_{\theta}$ берется комбинация вектора ребра и его выходной вершины:
$$(h_{\theta}(i,j))_m = ReLU(\theta_m (x_i - x_j) + \phi_m x_i)$$
В качестве функции агрегации взят максимум:
$$x'_{im} = \max\limits_{(i,j) \in E} ReLU(\theta_m (x_i - x_j) + \phi_m x_i)$$

Архитектура сети для сегментации состоит из преобразования полносвязным слоем, четырех слоев графовых сверток и серии полносвязных слоев, предсказывающих классы точек. Точность сегментации на датасете ModelNet40 составила 93.5\%, что выше, чем у остальных вышеперечисленных архитектур.

\subsection{Преимущества и недостатки}

Основным преимуществом графового метода обработки облаков точек является низкая ресурсоемкость. Так как структура графа не задает дополнительных размерностей, то для хранения сети и вычисления результата памяти нужно ненамного больше, чем для хранения исходного облака точек. Так, модель EdgeConv \cite{wang2019dynamic} занимает 21 Мб памяти, модель PointNet++ \cite{qi2017pointnet++} - 12 Мб. Скорость работы графовых моделей также достаточно высокая - 27 мс на видеокарте у модели EdgeConv и 163 мс у модели PointNet++.

Еще одно преимущество графового метода - его вариативность. Можно задать разные структуры графов и разные функции для свертки, что позволяет наилучшим образом адаптировать архитектуру нейронной сети под конкретную задачу.

Основным недостатком графового метода является трудоемкость реализации. Свертки на графах не реализованы в стандартных фреймворках для глубокого обучения, и их эффективная реализация для параллельных вычислений может занять немалое время.

\bigskip

\bibliography{ComputerVisionMIPT}
\bibliographystyle{plain}

\end{document}